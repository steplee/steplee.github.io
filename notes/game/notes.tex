\documentclass{article}

\author{Stephen Lee}
\title{Game Theory}
\usepackage{amsthm}
\usepackage{array}

\newtheorem{remark}{Remark}[section]
\newtheorem{defn}{Definition}[section]

%\def \lecture[#1]{\medbreak\refstepcounter{section}%
  %\renewcommand{\leftmark}{Lecture \thesection

\begin{document}

\maketitle

\section{Lecture 1: Intro}

\subsection{Prisoner's Dilemma}
\[ \begin{array}{l|c|c}
a & C & D \\
\hline
C & -1,-1 & -5, 0 \\
\hline
D & 0,-5 & -3,-3
\end{array}
\]

\noindent
Normal Form. \\
Pure Strategies. \\
Mixed Strategies. \\
Expected Utility.  \\

Utilities \& Rationality

\section{Lecture 2}
\subsection{}
People tend to be \emph{risk-averse}--they choose the less risky, even if the expected value is worse.

Anchoring: People latch onto (possibly irrelevant) information and it influences decisions. \\
\begin{defn}
  Positive affine: $u'(x)\quad=\quad c$
\end{defn}
Positive affine transformations preserv agent behaviour (equivalent games).
\begin{remark}
  Test
\end{remark}

Types of games:
\begin{itemize}
  \setlength\itemsep{0em}
  \item Common-payoff game
  \item Zero-sum game
    \begin{itemize}
      \item Matching Pennies
      \item Rock, Paper, Scissors
      \end{itemize}
  \item Nonzero-sum game
    \begin{itemize}
      \item Prisoner's Dillema
      \end{itemize}
    \item Symmetric game
\end{itemize}
Strategies:
\begin{it


\end{document}
