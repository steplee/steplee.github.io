\documentclass{article}

\author{Stephen Lee}
\title{Tao Notes}

\usepackage{amsthm}
\usepackage{amsmath}

%\theoremstyle{definition}
\newtheorem{definition}{Definition}[section]


\begin{document}

\maketitle

\section{Inverse Littlewood-Offord Theorems and Gromov-type}
\subsection{Inverse Theorems}
\begin{definition}{Inverse Theorem}
A description of objects with some combinatorial properties (i.e. a set that is bounded doubling, polynomial growth). 
\end{definition}
Seeks to give a most explicit definition than the input. Keeps as reversible as possible. \\
More examples: \begin{enumerate}
  \item Bounded Doubling ($\quad|A+A| \le k\,|A|\quad \Rightarrow %
    \exists \text{multinomial arithmetic progression P of rank r, for r} \le O_k(|A|), \text{for A} \subset P$)
\item 
\end{enumerate}




\end{document}

