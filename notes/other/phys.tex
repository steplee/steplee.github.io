\documentclass{article}

\usepackage{amsmath}

\author{Stephen Lee}
\title{Math Phys.}


\begin{document}

\maketitle

\section[4]{Lecture 4}

\paragraph{Shanks Transform (extrapolation)}
\paragraph{Richardson Extrapolation}


\paragraph{Euler Summation}


\paragraph{Borel Summation}

\paragraph{Generic Summation Machine}
\begin{align*} 
prop 1: & \qquad
S(a_0 + a_1 + \cdots) = a_0 + S(a_1 + a_2 + \cdots) \\
prop 2: & \qquad
S(\sum{\alpha a_n + \beta b_n}) = \alpha\,S(\sum{a_n}) + \beta\,S(\sum{b_n})
\end{align*}

\paragraph{\emph{Misc.}} 
There is a way of turning a divergent series into a convergent series (explained later). \\

Inside the ROC, a Taylor series is absolutely and uniformily convergent, and so order of summation doesn't matter.

\section{Lecture 11}
\paragraph{Herglatz}
Powerful condition, imag can be expressed as Fourier series. Sign of series same as sign of sin($\theta$). Recall $n\nsin(\theta)\quad\gesin(n\,\theta)$ 

\end{document}

